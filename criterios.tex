\documentclass[11pt]{scrartcl}
\usepackage[sexy]{evan}
\usepackage{amsthm}
\renewcommand{\proofname}{Prueba}
\usepackage{cancel}
\usepackage{xcolor}
\usepackage{answers}
\Newassociation{hint}{hintitem}{all-hints}
\renewcommand{\solutionextension}{out}
\renewenvironment{hintitem}[1]{\item[\bfseries #1.]}{}
\newcommand{\colorcancel}[2][black]{\renewcommand\CancelColor{\color{#1}}\cancel{#2}}
\title{Criterios de divisibilidad}


\author{AlanLG}
\date{Marzo 2024}

\begin{document}

\maketitle

\section{Lectura}
En esta lista trabajaremos propiedades elementales de teoría de números, la rama de las matemáticas que estudia el conjunto de los números enteros denotado por $\mathbb{Z}$:
\[\mathbb{Z}=\{\ldots,-2,-1,0,1,2,3,\ldots\}\]

Así como el conjunto de los naturales denotado por $\mathbb{N}$

\[\mathbb{N}=\{1,2,3,\ldots\}\]


\subsection{Divisibilidad}
Si $a$ y $b$ son dos número enteros ($a,b\in\mathbb{Z}$),decimos que $a$ divide a $b$ (ó bien $a\mid b$) si y solo si la división $\frac{a}{b}$ es entero, o bien
\[a\mid b \iff \frac{b}{a}\in \mathbb{Z}\]

También podemos decir que $a$ divide a $b$ si existe un entero $k$ tal que $b=a\cdot k$, o bien
\[a\mid b \iff \exists\hspace{0.1cm} k\in\mathbb{Z} : b=a\cdot k\]

Cuando $a$ \textbf{no divide} a $b$ lo denotamos por $a\nmid b$

\begin{ejemplo*}
    $7\mid 21$ pues $\frac{21}{7}=3\in \mathbb{Z}$
\end{ejemplo*}
\begin{ejemplo*}
    $20\mid 100$ pues $100=20\cdot(50)$
\end{ejemplo*}
\begin{ejemplo*}
    El 1 divide a todos los números (¿Por qué?)
\end{ejemplo*}
\subsection{Números primos}

\begin{proposition*}
[Definición de número primo]
    Decimos que un número entero positivo $p\neq \pm 1$ es primo si sus únicos divisores son $\pm1$ y $\pm p$
\end{proposition*}
\begin{definition*}
    Decimos que un número entero positivo diferente de $\pm 1$ es \textbf{Compuesto} si no es primo, a los número 1 y -1 se le llaman \textit{unidades}
\end{definition*}
\begin{ejemplo*}
    El número 31 es primo
\end{ejemplo*}
\begin{ejemplo*}
    El número 39 no es primo
\end{ejemplo*}

\begin{exercise*}
    Encuentra los primeros diez números primos
\end{exercise*}
\begin{ejemplo*}
    El único número primo que es par es el $2$ (¿Por qué?)
\end{ejemplo*}
\subsection{Criterios de divisibilidad}
En la siguiente lista introduciremos los criterios de divisibilidad, que sirven esencialmente para ver si un número divide a otro, muchos de ellos son intuitivos pero saberlos te puede facilitar un poco de cuentas.

\begin{criterio*}
    [Criterio de divisibilidad del 2]
    Que su último dígito sea divisible entre 2
\end{criterio*}
\begin{ejemplo*}
       $143\underline{4}$ es divisible entre 2 porque $4$ es divisible entre $2$
   \end{ejemplo*}

\begin{criterio*}
    [Criterio de divisibilidad del 3]
    Que la suma de sus dígitos sea divisible entre 3
\end{criterio*}
\begin{ejemplo*}
        2025 es divisible entre $3$ porque $2+0+2+5=9$ es divisible entre $3$
    \end{ejemplo*}
    \begin{ejemplo*}
        123456789 es divisible entre $3$ porque $1+2+3+4+5+6+7+8+9=45$ es divisible entre $3$
    \end{ejemplo*}
\begin{criterio*}
    [Criterio de divisibilidad del 4]
    Que el número formado por sus últimos dos dígitos sea divisible entre $4$
\end{criterio*}
\begin{ejemplo*}
    $333\underline{12}$ es divisible entre 4 porque 12 es divisible entre 4
\end{ejemplo*}
\begin{ejemplo*}
    $57\underline{04}$ es divisible entre 4 porque 4 es divisible entre 4
\end{ejemplo*}
\begin{criterio*}
    [Criterio de divisibilidad del 5]
    Que su último dígita sea $0$ ó $5$
\end{criterio*}
\begin{ejemplo*}
    $123\underline{0}$ es divisible entre 5 porque termina en 0
\end{ejemplo*}
\begin{ejemplo*}
    $343\underline{5}$ es divisible entre porque termina en 5
\end{ejemplo*}
\begin{criterio*}
    [Criterio de divisibilidad del 6]
    Que su último dígito sea par y la suma de sus dígitos sea divisible entre 3. Lo cúal es equivalente a que cumpla el criterio del 2 y del 3
\end{criterio*}
\begin{ejemplo*}
    12648 es divisible entre 6 porque termina en 8 y $1+2+6+4+8=21$ es divisible entre 3
\end{ejemplo*}

\begin{criterio*}
    [Criterio de divisibilidad del 8]
    Que el número formado por sus últimos 3 dígitos sea divisible entre 8
\end{criterio*}
\begin{ejemplo*}
    $45\underline{128}$ es divisible entre 8 porque 128 es divisible entre 8
\end{ejemplo*}
\begin{ejemplo*}
    $123\underline{000}$ es divisible entre 8 porque 0 es divisible entre 8
    \end{ejemplo*}
    \begin{ejemplo*}
    $2\overline{008}$ es divisible entre 8 porque 8 es divisible entre 8
\end{ejemplo*}
\begin{criterio*}
    [Criterio de divisibilidad del 9]
    Que la suma de sus dígitos sea divisible entre 9
\end{criterio*}
\begin{ejemplo*}
    126 es divisible entre 9 porque $1+2+6=9$ es divisible entre 9
\end{ejemplo*}
\begin{ejemplo*}
     123456789 es divisible entre 9 porque $1+2+3+4+5+6+7+8+9=45$ es divisible entre 9
\end{ejemplo*}
\begin{criterio*}
    [Criterio de divisibilidad del 10]
    Que su último dígito sea 0
\end{criterio*}
\begin{ejemplo*}
    $1000000\underline{0}$ es divisible entre 10 porque termina en 0
\end{ejemplo*}
\begin{ejemplo*}
    $3435\underline{0}$ es divisible entre 10 porque termina en 0
\end{ejemplo*}
\begin{criterio*}
    [Criterio de divisibilidad del 11]
    Que la suma de los dígitos en las posiciones pares menos la suma de los
dígitos en las posiciones impares sea múltiplo de 11.

\end{criterio*}
\begin{ejemplo*}
    8192734 es divisible entre 11 porque $(8+9+7+4)-(1+2+3)=22$ es divisible entre 11
\end{ejemplo*}
\begin{ejemplo*}
    132 es divisible entre $11$ porque $(3)-(1+2)=0$ es divisble entre 11
\end{ejemplo*}
\begin{ejemplo*}
    25366 es divisible entre 11 porque $(5+6)-(2+3+6)=0$ es divisible entre 11
\end{ejemplo*}
\section{Ejercicios}
\begin{exercise}
Si un número es divisible por 2 y 6, ¿podemos garantizar que es divisible por $2\times 6=12$
\end{exercise}
\begin{exercise}
    Si un número es divisible por 3 y por 4, ¿podemos garantizar que es divisible por $3\times 4=12$?
\end{exercise}
\begin{exercise}
Si un número es divisible por 4 y 6, ¿podemos garantizar que es divisible por $4\times 6=12$
\end{exercise}
\begin{exercise}
    La suma de dos números pares también es un número par
\end{exercise}
\begin{example}
    
\end{example}
\begin{example}
    Si sabemos que solo uno de los números 234, 2345, 23456, 234567, 2345678, 23456789 es primo. ¿Cuál de ellos es?

\end{example}
\textit{Solución.}Podemos ir descartando los números que no pueden ser primos, buscando algunos de sus divisores, primero descartemos los múltiplos de 2

\[\Large{\colorcancel[blue]{234}},\hspace{0.35cm} 2345, \hspace{0.35cm}\colorcancel[blue]{23456}, \hspace{0.35cm}234567, \hspace{0.35cm}\colorcancel[blue]{2345678}, \hspace{0.35cm}23456789 \]
Luego los múltiplos de $5$

\[\colorcancel[blue]{234}, \hspace{0.35cm}{\Large{\colorcancel[red]{2345}}}, \hspace{0.35cm}\colorcancel[blue]{23456},\hspace{0.35cm} 234567,\hspace{0.35cm} \colorcancel[blue]{2345678},\hspace{0.35cm} 23456789 \]
Nota que $3\mid 234567$ pues $3\mid 2+3+4+5+6+7=27$, entonces tambien lo podemos descartar

\[\colorcancel[blue]{234},\hspace{0.35cm} \colorcancel[red]{2345},\hspace{0.35cm} \colorcancel[blue]{23456},\hspace{0.35cm} {\Large{\colorcancel[green]{234567}}},\hspace{0.35cm} \colorcancel[blue]{2345678},\hspace{0.35cm} \boxed{23456789} \]

De modo que el primo buscado es 23456789

\begin{example}
    Encuentra el menor entero positivo $a$ que cumple que $a + 2a + 3a + 4a + 5a + 6a + 7a + 8a + 9a$ es un número con
todas sus cifras iguales
\end{example}
\textit{Solución.} Sea $N$ el número deseado; notemos que 
\[N=a + 2a + 3a + 4a + 5a + 6a + 7a + 8a + 9a=45a\]
Ahora nota que $5\mid N$ y $9\mid N$ (¿Por qué?)
Como 5 divide a $N$, sabemos que $N$ debe terminar en 0 o en 5, de modo que como, todas las cifras son iguales todos sus dígitos son 0 o 5, pero no pueden ser todos 0 (¿Por qué?). De modo que 

\[N=\underbrace{55\cdots 5}_{\text{$x$ cifras 5´s}}\]

Pero como $9\mid N$ entonces la suma de dígitos de $N$ es divisible entre $5$, pero la suma de dígitos de $N$ es $5x$ de modo que queremos hallar el menor $x\in\mathbb{Z}_{>0}$ tal que $9\mid 5x$, como $5$ no tiene factores en común con $5$ entonces $9\mid x$ de modo que el menor $x$ que cumple es $9$ y entonces

\[N=\underbrace{555555555}_{9}=45a \Rightarrow a=\frac{555555555}{45}=123456789\]

\bigskip
\section{Problemas}
\Opensolutionfile{all-hints}
\begin{problem}
h
    \begin{hint}
        vs
    \end{hint}
\end{problem}


\section{Hints}

\Closesolutionfile{all-hints}
\begin{enumerate}
  \input{all-hints.out}
\end{enumerate}


\end{document}