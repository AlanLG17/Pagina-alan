\documentclass[11pt]{scrartcl}
\usepackage[sexy]{evan}
\usepackage{amsthm}
\usepackage[helvetica]{quotchap} 

\title{3435}
\author{AlanLG}
\date{24 de Marzo del 2024}

\begin{document}

\maketitle

\epigraph{Los encantos de estsa ciencia sublime, las matemáticas, sólo se le revelan a aquellos que tienen en el valor de profundizar en ellas}
{\emph{Carl F. Gauss}}


\begin{problem}[\href{}{OMMEB 2021}]
    
    En siguiente figura se muestran un cuadrado, un pentágono y un hexágono. ¿Cuánto mide el angulo marcado?
    \begin{center}
 \begin{asy}
 /* Geogebra to Asymptote conversion, documentation at artofproblemsolving.com/Wiki go to User:Azjps/geogebra */
 import graph; size(6cm); 
real labelscalefactor = 0.5; /* changes label-to-point distance */
pen dps = linewidth(0.7) + fontsize(10); defaultpen(dps); /* default pen style */ 
pen dotstyle = black; /* point style */ 
real xmin = -10.492389149494306, xmax = 18.530442514124825, ymin = -3.380426703654242, ymax = 13.83607883896302;  /* image dimensions */
pen zzttqq = rgb(0.6,0.2,0); 

draw((-4,1)--(0,0)--(1,4)--(-3,5)--cycle, linewidth(0.4) + zzttqq); 
draw((-3,5)--(1,4)--(3.1871244937949434,7.495209070805666)--(0.5388417685876277,10.655367074350506)--(-3.2850114612046353,9.113243059555561)--cycle, linewidth(0.4) + zzttqq); 
draw((3.1871244937949434,7.495209070805666)--(1,4)--(2.9333775999580385,0.3582900917315648)--(7.053879693711021,0.21178925426879536)--(9.241004187505965,3.706998325074461)--(7.3076265875479285,7.348708233342896)--cycle, linewidth(0.4) + zzttqq); 
draw(arc((2.9333775999580385,0.3582900917315648),0.5056242450107863,141.96375653207355,186.96375653207355)--(2.9333775999580385,0.3582900917315648)--cycle, linewidth(2) + red); 
 /* draw figures */
draw((0,0)--(-4,1), linewidth(0.4)); 
draw((-4,1)--(0,0), linewidth(0.4) + zzttqq); 
draw((0,0)--(1,4), linewidth(0.4) + zzttqq); 
draw((1,4)--(-3,5), linewidth(0.4) + zzttqq); 
draw((-3,5)--(-4,1), linewidth(0.4) + zzttqq); 
draw((-3,5)--(1,4), linewidth(0.4) + zzttqq); 
draw((1,4)--(3.1871244937949434,7.495209070805666), linewidth(0.4) + zzttqq); 
draw((3.1871244937949434,7.495209070805666)--(0.5388417685876277,10.655367074350506), linewidth(0.4) + zzttqq); 
draw((0.5388417685876277,10.655367074350506)--(-3.2850114612046353,9.113243059555561), linewidth(0.4) + zzttqq); 
draw((-3.2850114612046353,9.113243059555561)--(-3,5), linewidth(0.4) + zzttqq); 
draw((3.1871244937949434,7.495209070805666)--(1,4), linewidth(0.4) + zzttqq); 
draw((1,4)--(2.9333775999580385,0.3582900917315648), linewidth(0.4) + zzttqq); 
draw((2.9333775999580385,0.3582900917315648)--(7.053879693711021,0.21178925426879536), linewidth(0.4) + zzttqq); 
draw((7.053879693711021,0.21178925426879536)--(9.241004187505965,3.706998325074461), linewidth(0.4) + zzttqq); 
draw((9.241004187505965,3.706998325074461)--(7.3076265875479285,7.348708233342896), linewidth(0.4) + zzttqq); 
draw((7.3076265875479285,7.348708233342896)--(3.1871244937949434,7.495209070805666), linewidth(0.4) + zzttqq); 
draw((0,0)--(2.9333775999580385,0.3582900917315648), linewidth(0.4)); 
draw((2.9333775999580385,0.3582900917315648)--(-3,5), linewidth(0.4)); 
 /* dots and labels */
dot((0,0),linewidth(1pt) + dotstyle); 
dot((-4,1),linewidth(1pt) + dotstyle); 
dot((1,4),linewidth(1pt) + dotstyle); 
dot((-3,5),linewidth(1pt) + dotstyle); 
dot((3.1871244937949434,7.495209070805666),linewidth(1pt) + dotstyle); 
dot((0.5388417685876277,10.655367074350506),linewidth(1pt) + dotstyle); 
dot((-3.2850114612046353,9.113243059555561),linewidth(1pt) + dotstyle); 
dot((2.9333775999580385,0.3582900917315648),linewidth(1pt) + dotstyle); 
dot((7.053879693711021,0.21178925426879536),linewidth(1pt) + dotstyle); 
dot((9.241004187505965,3.706998325074461),linewidth(1pt) + dotstyle); 
dot((7.3076265875479285,7.348708233342896),linewidth(1pt) + dotstyle); 
clip((xmin,ymin)--(xmin,ymax)--(xmax,ymax)--(xmax,ymin)--cycle); 
 /* end of picture */
 \end{asy}   
\end{center}

\end{problem}

\begin{problem}
   Encuentra el valor de la siguiente suma

   \[\frac{1}{1\cdot 2}+\frac{1}{2\cdot 3}+\frac{1}{3\cdot 4}+\cdots +\frac{1}{98\cdot 99}+\frac{1}{99\cdot 100}
\end{problem}

\begin{problem}
    [\href{https://artofproblemsolving.com/community/c6h3194066p29146566}{OMM 2024 }] 
    Encuentre los números enteros positivos de cuatro dígitos tales que la suma de los cuadrados de los dígitos sea igual al doble de la suma de los dígitos.

\begin{center}

\end{center}
\end{problem}

\begin{problem}
 
    [\href{https://artofproblemsolving.com/community/q1h3119244p28239057}{OMCC 2023 }] 
    Octavio escribe un número entero $n \geq 1$ en una pizarra y luego inicia un proceso en el que, en cada paso, borra el número entero $k$ escrito en la pizarra y lo reemplaza con uno de los siguientes números:
\[3k-1, \quad 2k+1, \quad \frac{k}{2}.\]
siempre que el resultado sea un número entero.

\item Demuestre que para cualquier número entero $n \geq 1$, Octavio puede escribir en la pizarra el número $3^{2023}$ después de un número finito de pasos.


\end{problem}

\end{document}