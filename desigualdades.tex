\documentclass[11pt]{scrartcl}
\usepackage[sexy]{evan}
\usepackage{amsthm}
\renewcommand{\proofname}{Prueba}


\usepackage{answers}
\Newassociation{hint}{hintitem}{all-hints}
\renewcommand{\solutionextension}{out}
\renewenvironment{hintitem}[1]{\item[\bfseries #1.]}{}
\raggedright

\title{desigualdades con esteroides}


\author{AlanLG}
\date{5 de abril del 2024}

\begin{document}

\maketitle

\section{Lectura}

Empezamos con problablemente el resultado más conocido de desigualdades, y es la desigualdad de medias, muchos problemas de desigualdades pueden ser atacados con esto de alguna manera, incluso problemas muy complejos.

\begin{theorem}
    [Desigualdad de medias]\label{Desigualdad de medias}
    Sean $x_1, x_2,\cdots x_n\in \mathbb{R^{+}}$, entonces
    \[\sqrt{\frac{x_1^2+x_2^2+\cdots x_n^2}{n}}\geq \frac{x_1+x_2+\cdots x_n}{n}\geq \sqrt[n]{x_1x_2\cdots x_n}\geq \frac{n}{\frac{1}{x_1}+\frac{1}{x_2}+\cdots+\frac{1}{x_n}}\]
\end{theorem}

Podemos generalizar un poco la famosa desigualdad de $AM-GM$ con el siguiente enunciado

\begin{theorem}
    [Weighted $AM-GM$]\label{Weighted AM-GM}

    Si $w_1+\cdots w_n=w$ y $x_i\in \mathbb{R^+}$, entonces 
    \[\frac{w_1x_1+\cdots w_nx_n}{w}\geq\sqrt[w]{x_1^{w_1}\cdots x_n^{w_n}}\]
Igualdad cuando $x_1=\cdots=x_n$ y $w_1=\cdots=w_n=1/n$.\\
En particular si $w_1+\cdots w_n=1$ entonces
    \[w_1x_1+\cdots w_nx_n\geq x_1^{w_1}\cdots x_n^{w_n}\]
\end{theorem}

Un problema famoso (y muy odiado) donde se usa este resultado es en el problema 2 de la IMO del 2020
\begin{example}
    [\href{https://artofproblemsolving.com/community/c6h2278647p17821569}{IMO 2020/2}]
    Sean $a,b,c,d$ números reales tales que $a\geq b\geq c\geq d>0$ y $a+b+c+d=1$. Prueba que \[(a+2b+3c+4d)a^ab^bc^cd^d<1\]
\end{example}
\begin{proof}[\href{https://artofproblemsolving.com/community/c6h2278647p17822092}{\text{Solución de Evan}}]
    La expresión $a^ab^bc^cd^d$ puede resultar muy rara, pero note que por $\hyperref[Weighted AM-GM]{\text{Weighted $AM-GM$}}$ tenemos que \[a^ab^bc^cd^d\leq a^2+b^2+c^2+d^2\]
    Nota que, como $a+b+c+d=1$, entonces acabamos si demostramos lo siguiente
    \[(a + 2b + 3c + 4d)(a^2 + b^2 + c^2 + d^2) <(a + b + c + d)^3\]

    De aquí solo queda expandir y simplificar términos, lo cuál es aún bastante largo, pero ya no se necesita de nada más, al expandir queda 

    \[ \begin{array}{cccc} +a^3 &+ b^2a &+ c^2a & +d^2a \\ +2a^2b &+ 2b^3 &+ 2b^2c & +2d^2b \\ +3a^2c & + 3b^2c & + 3c^3 & + 3d^2c \\ +4a^2d &+ 4b^2d & + 4c^2d & + 4d^3 \end{array} < \begin{array}{cccc} +a^3 &+ 3b^2a &+ 3c^2a & +3d^2a \\ +3a^2b &+ b^3 &+ 3b^2c & +3d^2b \\ +3a^2c &+ 3b^2c &+ c^3 &+ 3d^2c \\ +3a^2d &+ 3b^2d &+ 3c^2d &+ d^3 \\ +6abc &+ 6bcd &+ 6cda &+ 6dab \end{array} \]
    Entonces cancelamos algunos términos y queda 

    \[ \begin{array}{cccc} & && \\ &+ b^3 & & \\ & & +2c^3 & \\ +a^2d &+ b^2d & + c^2d & + 3d^3 \\ \end{array} < \begin{array}{cccc} &+ 2b^2a &+ 2c^2a & +2d^2a \\ +a^2b & &+ b^2c & +d^2b \\ &&& \\ &&& \\ +6abc &+ 6bcd &+ 6cda &+ 6dab \end{array} \]

    Lo cuál es cierto pues
    \[2b^2a \ge b^3 + c^2d \hspace{0.69cm} 2c^2a \ge 2c^3 \hspace{0.69cm} 2d^2a \ge 2d^3\hspace{0.69cm} a^2b \ge a^2d \hspace{0.69cm} b^2c \ge b^2d \hspace{0.69cm} d^2b \ge d^3\]
\end{proof}

Veamos ahora un famoso resultado llamado $\textit{Power Mean}$, es la versión generalizado de todas las desigualdades de medias; la primera vez que ví este resultado no podía creer que algo así existiera


\begin{theorem}
[Power Mean Inequality]\label{Power Mean Inequality}
Sean $a_1,a_2,\ldots,a_n$ y $w_1,w_2,\ldots w_n$ reales positivos con $w_1+w_2+\cdots+w_n=1$, para cada $r\in \mathbb{R}$ definimos 

\[\mathbb{M}_r
        \left(a_1,a_2,\dots a_n\right)=\begin{cases}\left(\frac{a_1^r+a_2^r+\cdots+ a_n^r}{n}\right)^\frac{1}{r} \hspace{0.2cm} \text{si} \hspace{0.2cm}r\neq 0\in\mathbb{R} \\ \sqrt[n]{a_1a_2\cdots a_n} \hspace{0.2cm} \text{si}\hspace{0.2cm} r=0 \end{cases}\]

        entonces si $p>q$, se tiene que
    \[ \mathbb{M}_p\geq \mathbb{M}_q\]

     Igualdad cuando $a_1=a_2=\cdots=a_n$


\end{theorem}


wow, realmente es un enunciado fácil de citar; pero si no te has dado cuenta de su utilidad, mira cuanto valen $\mathbb{M}_r$ para algunas $r$



\[ \begin{array}{ccccccc} 2& \geq&  1& \geq&  0 &\geq&  -1 \\ \\\mathbb{M}_2\left(x_1,\dots ,x_n\right) & \geq & \mathbb{M}_1\left(x_1,\dots ,x_n\right)&\geq  &\mathbb{M}_0\left(x_1,\dots ,x_n\right)&\geq &\mathbb{M}_{-1}\left(x_1,\dots ,x_n\right) \\ \\  \sqrt{\frac{x_1^2+x_2^2+\cdots x_n^2}{n}} &\geq & \frac{x_1+x_2+\cdots x_n}{n} & \geq &\sqrt[n]{x_1x_2\cdots x_n} &\geq & \frac{n}{\frac{1}{x_1}+\frac{1}{x_2}+\cdots+\frac{1}{x_n}} \end{array} \]

De modo que este teorema implica por si solo cada desigualdad de medias.\\

Pero por si fuera poco, también existe la \textbf{Weighted Power Mean Inequality} ({\small Estoy poniendo los nombres en inglés porque no encontré un nombre en español}), que es una mega generalización de las desigualdades de media, y se enuncia como sigue


\begin{theorem}
    [Weighted Power Mean Inequality]\label{Weighted Power Mean Inequality}

    Sea $a_1,a_2,\ldots,a_n$ reales positivos para cada $r\in \mathbb{R}$ definimos\[\mathbb{M}_r(a,w)
        =\begin{cases}\left(w_1a_1^r+w_2a_2^r+\cdots+w_na_n^r\right)^\frac{1}{r} \hspace{0.2cm} \text{si} \hspace{0.2cm}r\neq 0\in\mathbb{R} \\ a_1^{w_1}a_2^{w_2}\cdots a_n^{w_n}\hspace{0.2cm} \text{si}\hspace{0.2cm} r=0 \end{cases}\] entonces si $p>q$, se tiene que
    \[ \mathbb{M}_p\geq \mathbb{M}_q\]

    Igualdad cuando $a_1=a_2=\cdots=a_n$

    
\end{theorem}

Veamos ahora la desigualdad de Hölder, que es una versión fuerte de la famosa desigualdad de \textit{Cauchy-Schwartz}

\begin{theorem}[Desigualdad de Hölder]\label{Holder}
Para secuencias $a_i,b_i,\ldots, z_i$ y $\lambda_{a}+\lambda_{b}+\cdots+\lambda_{z}=1$ se tiene que

\[\left(\sum_{i=1}^n a_i\right)^{\lambda_{a}}\cdot \left(\sum_{i=1}^n b_i\right)^{\lambda_{b}}\cdots \left(\sum_{i=1}^n z_i\right)^{\lambda_{z}} \geq \sum_{i=1}^n a_i^{\lambda_{a}}b_i^{\lambda_{b}}\cdots z_i^{\lambda_{z}}\]

La igualdad se da si $a_1:a_2:\cdots :a_n=b_1:b_2:\cdots :b_n=\cdots=z_1:z_2:\cdots :z_n$

\end{theorem}
\begin{proof}
    Notemos que podemos asumir sin pérdida de la generalidad que $\sum a=\cdots \sum z=1$, de modo que basta probar que el lado derecho es menor que $1$, pero note que por $\hyperref[Weighted AM-GM]{\text{Weighted $AM-GM$}}$, tenemos que

    \[ \sum_{i=1}^n a_i^{\lambda_{a}}b_i^{\lambda_{b}}\cdots z_i^{\lambda_{z}} \leq \sum_{i=1}^n (\lambda_{a} a_i+\lambda_{b} b_i+\cdots+\lambda_{z} z_i)=1\]
    
\end{proof}

\begin{example}
    Prueba que \[9(a^3+b^{3}+c^{3})\geq (a+b+c)^3\]
\end{example}\begin{proof}
Por $\hyperref[Holder]{\text{Hölder}}$ nota que 
\[\left(a^3+b^3+c^3\right)^{\frac{1}{3}}\cdot \left(1+1+1\right)^{\frac{1}{3}}\cdot\left(1+1+1\right)^{\frac{1}{3}}\geq (a+b+c)\]
Elevando al cubo y reordenando obtenemos lo deseado
    
\end{proof}
\begin{remark*}
    La desigualdad anterior se puede probar fácilmente con $\hyperref[Weighted Power Mean Inequality]{\text{Weighted Power Mean}}$ usando que $\mathbb{M}_{3}\left(a+b+c\right)\geq\mathbb{M}_{1}\left(a+b+c\right)$
\end{remark*}

\begin{example}
    [\href{https://artofproblemsolving.com/community/c6h17451p119168}{IMO 2001/2}]
    Prueba que para todos reales positivos $a,b,c$ se cumple \[ \frac{a}{\sqrt{a^2 + 8bc}} + \frac{b}{\sqrt{b^2 + 8ca}} + \frac{c}{\sqrt{c^2 + 8ab}} \geq 1.  \]
\end{example}
\begin{proof}
Por $\hyperref[Holder]{\text{Hölder}}$ se tiene que
\[\left(\sum_{cyc} a(a^2+8bc)\right)^{\frac{1}{3}}\cdot \left(\sum_{cyc} \frac{a}{\sqrt{a^2+8bc}}\right)^{\frac{2}{3}}\geq(a+b+c)\]
Entonces basta probar que 
\begin{align*}
    (a+b+c)^3&\geq \sum_{cyc} a(a^2+8bc)=a^3+b^3+c^3+24abc \\ (a+b)(b+c)(c+a) &\geq 8abc
\end{align*}
Lo cuál se sigue por $\hyperref[Desigualdad de medias]{\text{AM-GM}}$
\end{proof}

Veamos por último un teorema fácil de citar pero realmente impresionante
\begin{theorem}
    [Desigualdad de Minkowski]\label{Minkowski}
    Definamos la norma $L_p$ para $p\geq 1$ como $||x||_p=\sqrt[p]{|x_1|^p+|x_2|^p+\cdots+|x_n|^p}$, entonces 
    \[||x||_p+||y||_p\geq ||x+y||_p\]
\end{theorem}


Esta desigualdad es como una versión fuerte de la desigualdad del triángulo. Veamos la demostración clásica de este teorema
\begin{proof}
Nota que 
\begin{align*}
    \sum_{k=1}^n(x_n+y_n)^p &=\sum_{k=1}^n x_k(x_k+y_k)^{p-1}+\sum_{k=1}^n y_k(x_k+y_k)^{p-1}\\
    &\stackrel{\hyperref[Holder]{\text{Hölder}}}{\leq}
 \left[\left(\sum_{k=1}^n x_i^p\right)^{\frac{1}{p}}\left(\sum_{k=1}^n (x_k+y_k)^p\right)^{\frac{p-1}{p}}\right]+\left[\left(\sum_{k=1}^n y_i^p\right)^{\frac{1}{p}}\left(\sum_{k=1}^n (x_k+y_k)^p\right)^{\frac{p-1}{p}}\right] \\ &=\left(\sum_{k=1}^n (x_k+y_k)^p\right)^{\frac{p-1}{p}}\left[\left(\sum_{k=1}^n x_i^p\right)^{\frac{1}{p}}+ \left(\sum_{k=1}^n y_i^p\right)^{\frac{1}{p}}\right]
\end{align*}
Reacomodando llegamos a

\[\left(\sum_{k=1}^n x_i^p\right)^{\frac{1}{p}}+ \left(\sum_{k=1}^n y_i^p\right)^{\frac{1}{p}}\geq \left(\sum_{k=1}^n(x_n+y_n)^p\right)^{\frac{1}{p}}\]

Que es precisamente lo deseado
\end{proof}
\begin{example}
    [\href{https://artofproblemsolving.com/community/c4h75015p431997}{AIME 1991}]

    Para un entero positivo $n$, define $S_n$ como el valor mínimo de la suma 
\[ \sum_{k=1}^n \sqrt{(2k-1)^2+a_k^2} \] 
donde $a_1,a_2,\ldots,a_n$ son números reales positivos cuya suma es 17. Existe un único entero positivo $n$ para el cual $S_n$ también es un número entero. Encuentra este valor de $n$.
\end{example}

   \textit{Solución.} Nota que por la desigualdad de $\hyperref[Minkowski]{\text{Minkowski}}$, tenemos que 

\begin{align*}
    \sum_{k=1}^n \sqrt{(2k-1)^2+a_k^2}&\geq \sqrt{\left(\sum_{k=1}^n (2k-1)\right)^2+\left(\sum_{k=1}^n a_k\right)^2}\\ &=\sqrt{n^4+17^2}
\end{align*} 
Solo queda ver cuándo $\sqrt{n^4+17^2}\in \mathbb{Z}$ lo cuál es sencillo.
\\
\section{Problemas}


\Opensolutionfile{all-hints}


\begin{problem}
En contruccción...

  \begin{hint}
  En contruccción...
  \end{hint}
\end{problem}






\bigskip

\section{Hints}
\Closesolutionfile{all-hints}
\begin{enumerate}
  \input{all-hints.out}
\end{enumerate}


\end{document}