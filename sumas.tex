 \documentclass[11pt]{scrartcl}
\usepackage[sexy]{evan}
\usepackage{amsthm}
\renewcommand{\proofname}{Prueba}
\usepackage{cancel}
\usepackage{xcolor}
\usepackage{answers}
\Newassociation{hint}{hintitem}{all-hints}
\renewcommand{\solutionextension}{out}
\renewenvironment{hintitem}[1]{\item[\bfseries #1.]}{}
\newcommand{\colorcancel}[2][black]{\renewcommand\CancelColor{\color{#1}}\cancel{#2}}
\title{series telescópicas}
\raggedright

\author{AlanLG}
\date{6 de mayo}

\begin{document}

\maketitle

\section{Lectura}
Al momento de evaluar una operación de varios términos, nos podemos apoyar en técnicas de cancelar algunos de estos, es aquí donde aparacen las series telescópicas, no hay mucha teoría en esto, se trata de hacer una adecuacda manipulación algebraica, resolver varios problemas te dará una idea de que se debe hacer en cada caso. Veamos algunos ejemplos

\begin{example}
    Halle el valor del producto

$$\left(1-\frac{1}{4}\right)\left(1-\frac{1}{9}\right)\left(1-\frac{1}{16}\right)\cdots\left(1-\frac{1}{81}\right)$$

\end{example}

\begin{walkthrough} 
 Hacer individualmente cada producto sería muy cansado, asi que vamos a pensar en algo más inteligente
 \begin{walk}
    \ii  !Factoriza!  
    \ii Trata de cancelar numeradores y denominadores de forma inteligente
\end{walk}
\end{walkthrough}

\begin{example}
    Encuentra el valor de 
    \[\frac{1}{1\cdot 2}+\frac{1}{2\cdot 3}+\frac{1}{3\cdot 4}+\cdots+\frac{1}{99\cdot 100}\]
  
\end{example}

\begin{walkthrough}
    \begin{walk}
        \ii Nota que el término general es $\frac{1}{k(k+1)}$
        \ii $\frac{1}{k(k+1)}=\frac{(k+1)-k}{k(k+1)}$
        \ii $\frac{(k+1)-k}{k(k+1)}=\frac{(k+1)}{k(k+1)}-\frac{k}{(k(k+1)}=\frac{1}{k}-\frac{1}{k+1}$
        \ii Sustituye para cada término y la suma te queda
        \[\left(\frac{1}{1}-\frac{1}{2}\right)+\left(\frac{1}{2}-\frac{1}{3}\right)+\left(\frac{1}{3}-\frac{1}{4}\right)+\cdots+\left(\frac{1}{99}-\frac{1}{100}\right)\]
        \ii Casi todos los término se cancelan. ¿Cuáles quedan?
    \end{walk}
\end{walkthrough}

\bigskip
\section{Problemas}
\Opensolutionfile{all-hints}

\begin{problem}
    Encuentra el valor de 
    \[\frac{1}{1+2}+\frac{1}{1+2+3}+\frac{1}{1+2+3+4}+\cdots+\frac{1}{1+2+3+\cdots+2024}\]
    \begin{hint}
        Cada término queda como $\frac{2(n+1-n)}{n(n+1)}$
    \end{hint}
\end{problem} \hspace{0.2cm}
\begin{problem}
    Encuentra el valor de 
    \[\frac{1}{1\times 3}+\frac{1}{3\times 5}+\frac{1}{5\times 7}+\cdots+\frac{1}{2021\times 2023}\]
    \begin{hint}
        $\frac{1}{n(n+2)}=\frac{1}{2}\left(\frac{n+2-n}{(n+2)n}\right)$
    \end{hint}
\end{problem} \hspace{0.2cm}
\begin{problem}
    Calcula
    \[\frac{1}{\sqrt{1}+\sqrt{2}}+\frac{1}{\sqrt{2}+\sqrt{3}}+\cdots+\frac{1}{\sqrt{99}+\sqrt{100}}\]
    \begin{hint}
        Multiplica a cada término por $\frac{\sqrt{n+1}-\sqrt{n}}{\sqrt{n+1}-\sqrt{n}}$
        
    \end{hint}
    \end{problem} \hspace{0.2cm}
    \begin{problem}
        Evalua
        \[\sum_{n=1}^{2023} \frac{1}{n^2+3n+2}\]
        \begin{hint}
            $\frac{1}{n^2+3n+2}=\frac{(n+2)-(n+1)}{(n+2)(n+1)}$
        \end{hint}
    \end{problem} \hspace{0.2cm}
    \begin{problem}
        Encuentra el valor de 
        \[1\cdot 1!+2\cdot 2!+3\cdot 3!+\cdots + 2023\cdots 2023!\]
        \begin{hint}
            Nota que $(k+1)!=(k+1)k!$
        \end{hint}
    \end{problem} \hspace{0.2cm}
    
    \begin{problem}
        Demuestra que 
        \[\sum_{k=1}^n\frac{1}{F_{k-1}F_{k}}<1\]
        para todo $n$. Donde $F_n$ representan a los términos de la sucesion de Fibonacci, con $F_0=1, F_1=1$ y $F_{n+2}=F_{n+1}+F_{n}$
        \begin{hint}
            $\frac{1}{F_{k-1}F_k}=\frac{F_{k+1}-F_k}{F_k}\cdot \frac{1}{F_{k-1}F_k}$
        \end{hint}
    \end{problem} \hspace{0.2cm}
    \begin{problem}
        Calcula el valor de 
        \[\frac{1}{1\sqrt{2}+2\sqrt{1}}+\frac{1}{2\sqrt{3}+3\sqrt{2}}+\frac{1}{3\sqrt{4}+4\sqrt{3}}+\cdots+\frac{1}{2023\sqrt{2024}+2024\sqrt{2023}}\]
        \begin{hint}
            Nota que $\frac{1}{n\sqrt{n+1}+(n+1)\sqrt{n}}=\frac{1}{\sqrt{n(n+1)}}\cdot\left(\frac{1}{\sqrt{n}+\sqrt{n+1}}\right)$, ¿A qué se parece el segundo término?
        \end{hint}
        \end{problem}
        \begin{problem}
            [\href{https://artofproblemsolving.com/community/c5h1784695p11777930}{2019 AMC12}]
            Sea $f(x) = x^{2}(1-x)^{2}$. ¿Cuánto vale la siguiente suma?
\[f\left(\frac{1}{2019}\right)-f\left(\frac{2}{2019}\right)+f\left(\frac{3}{2019}\right)-f\left(\frac{4}{2019}\right)+\cdots
+f\left(\frac{2017}{2019}\right) - f\left(\frac{2018}{2019}\right)\]
\begin{hint}
    ¿Cuánto vale $f(x)-f(1-x)$
\end{hint}
\end{problem} \hspace{0.2cm}
\begin{problem}
    [\href{https://artofproblemsolving.com/community/c1068820h2027275p14272522}{1999 Estonia MO}] Dado $f(x)=\frac{x^2}{1+x^2}$, encuentra el valor de 
\[f\left( \frac{1}{2000} \right)+f\left( \frac{2}{2000} \right)+\cdots+ f\left( \frac{1999}{2000} \right)+f\left( \frac{2000}{2000} \right)+f\left( \frac{2000}{1999} \right)+\cdots+f\left( \frac{2000}{1} \right)\]
\begin{hint}
    ¿Cuánto vale $f(x)+f\left(\frac{1}{x})\right)$?
\end{hint}
        \end{problem} \hspace{0.2cm}
   \begin{problem}
       Encuentra el valor de \[\left(1+\frac{1}{2}\right)\left(1+\frac{1}{2^2}\right)\left(1+\frac{1}{2^3}\right)\cdots \left(1+\frac{1}{2^{2^n}}\right)\]
       \begin{hint}
           Mira que pasa si multiplicas por $\left(1-\frac{1}{2}\right)$
       \end{hint}
   \end{problem}
   \hspace{0.2cm}
    \begin{problem}
        Calcula el valor de
        \[\frac{1}{\sqrt{1+\sqrt{1^2-1}}}+\frac{1}{\sqrt{2+\sqrt{2^2-1}}}+\frac{1}{\sqrt{3+\sqrt{3^2-1}}}+\cdots+\frac{1}{\sqrt{99+\sqrt{99^2-1}}}\]
        \begin{hint}
            Multiplica cada término por $\frac{\sqrt{n+\sqrt{n^2-1}}}{\sqrt{n+\sqrt{n^2-1}}}$.  ¿Cuánto vale $\frac{(\sqrt{n-1}+\sqrt{n+1})^2}{2}$
        \end{hint}
    \end{problem} \hspace{0.2cm}
    \begin{problem}
        Evalúa la suma
        \[\sum_{k=1}^{9999}\frac{1}{\left(\sqrt{n}+\sqrt{n+1}\right)\left(\sqrt[4]{n}+\sqrt[4]{n+1}\right)}\]
        \begin{hint}
            Trata de racionalizar ambos productos
        \end{hint}
    \end{problem} \hspace{0.2cm}
   
    \begin{problem}
     [\href{https://artofproblemsolving.com/community/c4h496010p2785243}{1999 USAMTS}]   Encuentra el valor de la expresión
        \[\sqrt{1+\frac{1}{1^2}+\frac{1}{2^2}
        }+\sqrt{1+\frac{1}{2^2}+\frac{1}{3^2}
        }+\sqrt{1+\frac{1}{3^2}+\frac{1}{4^2}
        }+\cdots+\sqrt{1+\frac{1}{2023^2}+\frac{1}{2024^2}
        }\]
        \begin{hint}
            Demuestra que $1+\frac{1}{n^2}+\frac{1}{(n+1)^2}=\frac{(n(n+1)+1)^2}{n^2(n+1)^2}$
        \end{hint}
    \end{problem}
    \hspace{0.2cm}
    \begin{problem}
        [\href{https://math.stackexchange.com/questions/3123586/question-from-the-2011-imc-international-mathematics-competition-key-stage-iii?noredirect=1}{2011 IMC}]
        Las raíces de $x^2-2x-a^2-a=0$ son $\left(\alpha_1,\beta_1\right), \left(\alpha_2,\beta_2\right), \left(\alpha_3,\beta_3\right),\ldots \left(\alpha_{2010},\beta_{2010}\right),  \left(\alpha_{2011},\beta_{2011}\right)$ cuando $a=1,2,\ldots, 2011$ respectivamente. Evalúa
        \[\frac{1}{\alpha_1}+\frac{1}{\beta_1}+\frac{1}{\alpha_2}+\frac{1}{\beta_2}+\frac{1}{\alpha_3}+\frac{1}{\beta_3}+\cdots+\frac{1}{\alpha_{2010}}+\frac{1}{\beta_{2010}}+\frac{1}{\alpha_{2011}}+\frac{1}{\beta_{2011}}\]
    \end{problem}
     \begin{problem}
        Evalúa la suma 
        \[\sum_{n=1}^{10}\frac{n}{n^4+n^2+1}\]
        \begin{hint}
            Nota que $n^4+n^2+1=(n^2-n+1)(n^2+n+1)$, busca una manera de convertirlo en una suma telescópica, (analiza algunos valores de $n^2-n+1$ y $n^2+n+1$ para darte una idea)
        \end{hint}
    \end{problem} \hspace{0.2cm}

    \begin{problem}
        Evalúa el producto
        \[\prod_{n=2}^{20} \left(\frac{n^3-1}{n^3+1}\right)\]
        \begin{hint}
            Es similar al problema anterior, factoriza $n^3-1$ y $n^3+1$ y trata de probar que se cancelan varios términos
        \end{hint}
    \end{problem}\hspace{0.2cm}
    
\begin{problem}
    Encuentra el valor de \[\left \lfloor \frac{1}{\sqrt{2}}+\frac{1}{\sqrt{3}}+\frac{1}{\sqrt{4}}+\cdots+\frac{1}{100000}\right \lfloor\]
    \begin{hint}
        Demuestra que $2(\sqrt{k+1}-\sqrt{k})<\frac{1}{\sqrt{k}}<2(\sqrt{k}-\sqrt{k-1})$ y suma todas las desigualdad para cada $k$
    \end{hint}
\end{problem}
\bigskip
\clearpage
\section{Hints}
\Closesolutionfile{all-hints}
\begin{enumerate}
  \input{all-hints.out}
\end{enumerate}



\end{document}
