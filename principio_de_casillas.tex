\documentclass[11pt]{scrartcl}
\usepackage[sexy]{evan}
\usepackage{amsthm}
\renewcommand{\proofname}{Prueba}


\usepackage{answers}
\Newassociation{hint}{hintitem}{all-hints}
\renewcommand{\solutionextension}{out}
\renewenvironment{hintitem}[1]{\item[\bfseries #1.]}{}

\title{Principio de casillas}


\author{AlanLG}
\date{Febrero 2024}

\begin{document}

\maketitle

\section{Lectura}
Principio de casillas es algo muy intuitivo pero que puede ser muy útil al momento de resolver algunos problemas de matemáticas, vamos a comenzar con un simple ejemplo 
\begin{example}
    Tienes un saco con manzanas rojas y manzanas verdes, ¿Cúal es el mínimo número de manzanas que debes sacar de la bolsa para gantizar que sacaste dos de un mismo color?
\end{example}
\begin{flushleft}
    

Si sacas $2$ manzanas puede que ambas sean de distinto color pero al momento de sacar una más a fuerzas será de algún color que ya se repitió, de modo que necesitas $3$ manzanas

Así funciona el principio de casillas, en particular de este ejemplo podemos ya  sacar una versión del principio de casillas bastante intuitiva
\end{flushleft}
\begin{theorem}
    [Caso particular del principio de casillas]
    Si tenemos $n$ casillas y al menos $n+1$ objetos, habrá al menos una casilla con al menos $2$ objetos
\end{theorem}
\begin{flushleft}
Al igual que antes, esto es obvio, pues el "peor caso" sería que todos las casilla haya un objeto, necesitando $n$ objetos pero el siguiente objeto ya tendrá que estar en una caja con un objeto, usando ya $2$ objetos.
\end{flushleft}
Presentaremos ahora, la forma general del principio de casillas
\begin{theorem}
    [Principio de casillas]
    Si tenemos $n$ casillas y al menos $nk+1$ objetos, habrá al menos una casilla con al menos $k+1$ objetos
\end{theorem}
\begin{flushleft}
    
Trata de demostrar esto, la demostración no es muy diferente a la versión que ya vimos, pero quiero que estes convencido de que es cierto.

\begin{theorem}
    [Otra forma de casillas]
    Si $a_1,a_2,\ldots a_n$ son números positivos y $a$ es el promedio de estos entonces existe un $a_j$ y un $a_i$ tal que
    \[a_j\leq a\]
    \[a_i\geq a\]
\end{theorem}
Esto es bastante intuitivo (demuestrálo) y puede llegar a ser útil. Veamos algunos ejemplos donde se use el principio de casillas

\begin{example}
    De cinco puntos dentro o sobre los lados de un triángulo equilátero de lado $2$ hay dos puntos cuya distancia entre ellos en menor o igual a $1$
\end{example}
\begin{proof}
    En este ejemplo nosotros vamos a construir las casillas, vamos a dividir el triángulo equilátero en cuatro triángulos equiláteros de lado $1$ como se muestra
    \begin{center}
        
    \begin{asy}
        import graph; size(4cm); 
        draw((0,0)--(2,0)--(1,1.73205)--cycle);
        draw((0.5,0.86602)--(1,0)--(1.5,0.86602)--cycle);
        dot((0.3,0.4));
        dot((0.8,0.6));
        dot((1.2,1.1));
        dot((1,1.6));
        dot((1.3,0.3));
        
    \end{asy}
    \end{center}
    Son cuatro regiones como queremos colocar cinco puntos por el principio de casillas deben haber dos puntos en alguna región, o sea, hay dos puntos dentro de un triángulo equilátero de lado $1$, entonces esos dos puntos están a distancia menor o igual a $1$
\end{proof}
\end{flushleft}
\begin{example}
     Cuál es la máxima cantidad de reyes (de ajedrez) que puedes poner en un tablero de $8\times 8$
 de modo que no se ataquen entre sí?
\end{example}
\begin{walkthrough}
    \begin{walk}
        \ii Haz un acomodo en el que se entren $16$ reyes
        \ii Considera el siguiente acomodo
        \begin{center}
            \begin{asy}
                size(4.5cm);
                for (int i = 0; i <= 8; ++i) {
    draw((i,0)--(i,8));
    draw((0,i)--(8,i));
    }
    for (int i = 0; i <= 4; ++i) {
    draw((2i,0)--(2i,8),linewidth(2));
    draw((0,2i)--(8,2i),linewidth(2));
    }
            \end{asy}
        \end{center}
        \ii Muestra por qué no se pueden poner $17$ reyes
    \end{walk}
\end{walkthrough}
\begin{example}
    Si se eligen cinco números de los enteros del 1 al 8, demuestra que dos de ellos deben sumar $9$

\end{example}

\begin{proof}
    La suma de parejas que suman $9$ son
    \begin{center}
        \boxed{1,8}

        \boxed{2,7}

        \boxed{3,6}

        \boxed{4,5}
    \end{center}
    Esas serán nuestras casillas, y a los $5$ números que eligamos los colocaremos en su respectiva casilla; como vamos a escoger $5$ números y hay $4$ casillas entonces va a haber una casilla con dos números o sea habrá una pareja que suma $9$
\end{proof}
\begin{example}
    Demuestra que en una fiesta de $n$ personas siempre hay dos personas conocen a el mismo número de personas dentro de la fiesta
\end{example}
\begin{proof}
    Los conocidos de una persona dentro de la fiesta es un número dentro de $\{0,1,2,\ldots,n-1\}$ pero note que si hay una persona que no conoce a nadie (0 conocidos) entonces no puede haber alguien que conozca a todos ($n-1$ conocidos) de modo que hay $n-1$ números posibles que alguien puede tener de conocidos, pero hay $n$ personas entonces hay dos personas que conocen a la misma cantidad de persona
\end{proof}
En este caso nuestras casillas fueran el número de conocidos que puede tener alguien, y los objetos, las $n$ personas.
\begin{example}
   Prueba que si se colocan $26$ puntos dentro de un cuadrado de lado $1$ entonces hay $2$ a una distancia menor que $\frac{2}{7}$
\end{example}
En este ejemplo es complicado saber donde puedes empezar, sin embargo escoger de forma correcta nuestras casillas nos puede ayudar mucho.
\begin{proof}
    Comenzamos dividiendo al cuadrado de lado $1$ en una cuadrícula de $5\times 5$, como se muestra
    \begin{center}
        \begin{asy}
        size(5cm);

// Dibuja la cuadrícula
for (int i = 0; i <= 5; ++i) {
    draw((i,0)--(i,5), gray);
    draw((0,i)--(5,i), gray);
}

// Dibuja los puntos en el cuadrito (2,3)
dot((2.5,3.2));
dot((2.2,3.6));


// Opcional: dibuja el cuadrito (2,3)
pen borderPen = linewidth(1.2) + red;
pair A = (2,3), B = (3,3), C = (3,4), D = (2,4);
draw(A--B--C--D--cycle, borderPen);

// Ajusta el tamaño de la visualización
clip((0,0),(5,5));
        \end{asy}
    \end{center}
    Entonces en particular hay dos puntos que estan dentro de un cuadrado de lado $\frac{1}{5}$ entonces basta probar que la diagonal de este cuadrado mide menos que $\frac{2}{7}$, o sea que 
    \[\sqrt{\left(\frac{1}{5}\right)^2+\left(\frac{1}{5}\right)^2}\leq \frac{2}{7}\]
    Lo cuál es sencillo de demostrar
\end{proof}
\begin{example}
    Demuestra que un tablero de ajedrez de $8\times 8$ se colocan $17$ torres, demuestra que hay al menos $3$ torres que no se atacan entre sí
\end{example}
Tiene sentido pensar en dividir de alguna forma el tablero de $8\times 8$ en conjuntos de modo que entre ellos no se atacan.
\begin{proof}
    Considera la siguiente enumeración de las casillas del tablero de $8\times 8$
    \begin{center}
        $$\begin{array}{|c|c|c|c|c|c|c|c|} \hline
1&2&3&4&5&6&7&8\\ \hline
2&3&4&5&6&7&8&1\\ \hline
3&4&5&6&7&8&1&2\\ \hline
4&5&6&7&8&1&2&3\\ \hline
5&6&7&8&1&2&3&4\\ \hline
6&7&8&1&2&3&4&5\\ \hline
7&8&1&2&3&4&5&6\\ \hline
8&1&2&3&4&5&6&7\\ \hline
\end{array}$$
    \end{center}
    Nota que si dos si hay dos torres en casillas con el mismo número entonces estas nunca se atacan, como queremos poner $17$ torres (objetos) y hay $8$ números (casillas) entonces hay $3$ torres que están en casillas con el mismo número.
\end{proof}
\begin{example}
    Prueba que existe algún número con solo cifras $1$ y $3$, y al menos una de cada una, tal que sea múltiplo de $2023$
\end{example}
\begin{walkthrough} Si hay, vamos a construir una forma de hayarlos
    \begin{walk}
        \ii Asume, por contradicción, que no hay y considera los siguientes $2023$ números
        \[13,\hspace{0.2cm} 1313,\hspace{0.2cm} 131313,\hspace{0.2cm} \ldots ,\hspace{0.2cm} \underbrace{1313\cdots 13}_{2023}\]
        \ii Hay dos con el mismo residuo al dividirse entre $2023$ (¿Por qué?)
        \ii ¿Qué te queda al restar esos dos números?
        \ii Concluye
    \end{walk}
\end{walkthrough}
\section{Ejercicios}
\begin{exercise}
    Prueba que en un grupo de $13$ personas hay dos que nacieron el mismo mes
\end{exercise}
\begin{exercise}
    Si tengo $17$ peras, $13$ manzanas y $11$ naranjas dentro de un saco, ¿Cuál es el mínimo número de frutas que debo de sacar para garantizar que tengo una fruta de cada una?
\end{exercise}
\begin{exercise}
    Dados $12$ enteros, prueba que se pueden escoger $2$ de tal forma que su diferencia sea divisible entre $11$
\end{exercise}
\begin{exercise}
    Prueba que una línea recta que no pasa por uno de los vértices de un triángulo, no puede cortar los tres lados del triángulo.
\end{exercise}
\begin{exercise}
    Se tienen pelotas varias pelotas azules, rojas y verdes en un saco, ¿Cuál es el mínimo número de pelotas que debes sacar para garantizar que hay $10$ del mismo color?
\end{exercise}
\begin{exercise}
    Se colocan $5$ puntos dentor de un cuadrado de lado $2$. Demuestra que existen dos puntos que están a una distancia
    de a lo más $\sqrt{2}$
\end{exercise}
\section{Problemas}


\Opensolutionfile{all-hints}


\begin{problem}
Se eligen $n+1$ números dentro del conjunto $\{1,2,3\ldots, 2n\}$, demuestra que existen dos cuyo máximo común divisor es $1$

  \begin{hint}
  Divide el conjunto en $n$ conjuntos de modo que cada conjunto tenga dos elementos y su máximo común divisor sea $1$
  \end{hint}
\end{problem}
\vspace{0.1cm}
\begin{problem}
   Con los vértices de una cuadricula de $6\times 9$, se forman $24$ triangulos. Muestre que hay dos triángulos que tienen un vértice en común.
   \begin{hint}
       Cuántos vértices ocupan los $24$ triángulos
   \end{hint}
\end{problem}
\vspace{0.1cm}
\begin{problem}
     $51$ hombres y $49$ mujeres se acomodan en una mesa redonda. Prueba que existen dos hombres que están sentados diametralmente opuestos.
\begin{hint}
    Hay 50 parejas de personas que están sentadas diametralmente opuestas
\end{hint}
\end{problem}
\vspace{0.1cm}
\begin{problem}
    Demuestra que de $12$ numeros distintos de dos digitos, siempre hay dos cuya diferencia es un número de dos dígitos
    \begin{hint}
        La condición es equivalente a que su diferencia sea múltiplo de $11$(¿Por qué?)
    \end{hint}
\end{problem}
\vspace{0.1cm}
\begin{problem}
    Probar que si cada punto del plano se colorea de rojo o azul entonces existe un segmento de
longitud 1 cuyos extremos son del mismo color.
\begin{hint}
    Considerate los vértices de un triángulo equilátero de lado $1$
\end{hint}
\end{problem}
\vspace{0.1cm}
\begin{problem}
    Diez niños juntaron $40$ naranjas, demuestra que hay dos niños que juntaron la misma cantidad de narajanjas
    \begin{hint}
       \[0+1+2+3+4+5+6+7+8+9=45\]
    \end{hint}
\end{problem}
\vspace{0.1cm}
\begin{problem}
 Se colorean todos los puntos del plano de rojo o azul. Demuestra que existen cuatro puntos del mismo color que forman un rectángulo
\begin{hint}
    Asume que no se puede, considerate tres puntos en una misma línea del mismo color (¿Por qué existen?) y luego dibuja perpendiculares a esa recta con vértices en esos puntos, analiza las colores de los puntos de las paralelas a la primer línea
    \begin{center}
        \begin{asy}
             size(6cm);
            draw((0,0)--(2,0));
            draw((0,0)--(0,2));
            draw((1,0)--(1,2));
            draw((2,0)--(2,2));
            filldraw(circle((0,0),0.1), blue);
            filldraw(circle((1,0),0.1), blue);
            filldraw(circle((2,0),0.1), blue);
            draw((0,0.4)--(2,0.4), dashed);
            draw((0,0.8)--(2,0.8), dashed);
            draw((0,1.2)--(2,1.2), dashed);
            draw((0,1.6)--(2,1.6), dashed);
        \end{asy}
    \end{center}
\end{hint}
\end{problem}
\vspace{0.1cm}
\begin{problem}
Demuestra que un triángulo equilátero de lado uno no puede ser cubierto totalmente por triángulos equiláteros de lados menor que $1$
    \begin{hint}
       Uno de esos triángulos equiláteros con lado menor que $1$ no puede cubrir simultáneamente dos vértices del otro triángulo
    \end{hint}
\end{problem}
\vspace{0.1cm}
\begin{problem}
Sean $a, b, c$ y $d$ enteros, muestre que $(a-b)(a-c)(a-d)(b-c)(b-d)(c-d)$ es
divisible entre $12$.

    \begin{hint}
        Dentro de $4$ números hay $3$ con la misma paridad y hay $2$ que tienen el mismo residuo al dividirse entre $4$
    \end{hint}
\end{problem}
\vspace{0.1cm}
\begin{problem}
    Si $a_1, a_2, a_3,\ldots a_n$ es una permutación de los números $1,2,3,\ldots n$, demuestra que si $n$ es impar, entonces el siguiente producto es par
 \[(a_1 -1)(a_2 -2)(a_3 -3)\cdots (a_n-n)\]
    \begin{hint}
        La diferencia de dos números es par si tienen la misma paridad
    \end{hint}
\end{problem}
\vspace{0.1cm}
\begin{problem}
    Se tienen $5$ puntos con coordenadas enteras en el plano, demuestra que existen dos de ellos de tal forma que su punto medio también tiene coordenadas enteras
    \begin{hint}
        Dos puntos $(x_1,y_1), (x_2,y_2)$ tienen coordenadas enteras si $x_1,x_2$ y $y_1,y_2$ tienen la misma paridad (¿Por qué?) 
    \end{hint}
\end{problem}
\vspace{0.1cm}
\begin{problem}
    Demuestra que si se colorean los lados y diagonales de un hexágono regular de colores azul y rojo entonces hay un triángulo cuyos vértices son vértices del hexágono y sus lados son de un mismo color.
    \begin{hint}
        Demuestra que de un vértices salen tres lados que son del mismo color.
        \begin{center}
            \begin{asy}
               /* Geogebra to Asymptote conversion, documentation at artofproblemsolving.com/Wiki go to User:Azjps/geogebra */
import graph; size(5cm); 
real labelscalefactor = 0.5; /* changes label-to-point distance */
pen dps = linewidth(0.7) + fontsize(10); defaultpen(dps); /* default pen style */ 
pen dotstyle = black; /* point style */ 
real xmin = -3.425597295141094, xmax = 5.344895552817108, ymin = -2.7154597405403655, ymax = 6.174766798174948;  /* image dimensions */

 /* draw figures */
draw((-1.5,2.5980762113533187)--(3,5.196152422706633), linewidth(2) + red); 
draw((-1.5,2.5980762113533187)--(4.5,2.598076211353316), linewidth(2) + red); 
draw((-1.5,2.5980762113533187)--(0,0), linewidth(2) + red); 
draw((0,0)--(4.5,2.598076211353316), linewidth(1.2) + dotted); 
draw((4.5,2.598076211353316)--(3,5.196152422706633), linewidth(1.2) + dotted); 
draw((3,5.196152422706633)--(0,0), linewidth(1.2) + dotted); 
draw((-1.5,2.5980762113533187)--(0,5.196152422706633), linewidth(0.1)); 
draw((0,5.196152422706633)--(3,5.196152422706632), linewidth(0.1)); 
draw((0,0)--(3,0), linewidth(0.1)); 
draw((3,0)--(4.5,2.598076211353316), linewidth(0.1)); 
 /* dots and labels */
dot((-1.5,2.5980762113533187),linewidth(4pt) + dotstyle); 
clip((xmin,ymin)--(xmin,ymax)--(xmax,ymax)--(xmax,ymin)--cycle); 
 /* end of picture */
            \end{asy}
        \end{center}
    \end{hint}
\end{problem}
\begin{problem}
   Demuestra que si se escogen $8$ números distintos dentro del conjunto $\{1,2,3,\ldots,15\}$, existen tres parejas de ellos que tiene la misma diferencia positiva
   \begin{hint}
       La mayor diferencia es $14$ que solo ocurre una vez (¿Por qué?), ¿Cuantás parejas de números distintos se pueden formar?
   \end{hint}
\end{problem}
\vspace{0.1cm}

\begin{problem}
[\href{https://artofproblemsolving.com/community/c6h1507145p8926278}{Bundeswettbewerb Mathematik 2017}]
    En un polígono regular de $35$ lados, se colorean $15$ vértices están de rojo. Demuestra que siempre hay tres vértices rojos que forman un triángulo isósceles?
\begin{hint}
    Un polígono de $35$ contiene a $7$ pentágonos regulares que no tienen vértices en común
\end{hint}
\end{problem}
\vspace{0.1cm}
\begin{problem}
   
    Dados $7$ puntos dentro del interior de un círculo de radio 1, demuestra que hay dos que están a distancia menor que $1$, ¿se podrá con $6$ puntos?
\begin{hint}
    Divide el circulo en $6$ regiones, si hay $7$ puntos entonces usa casillas; para el caso de $6$ puntos, divide al circulo en $6$ regiones pero haz tu división de modo que un diametro pasé por algún punto de los $6$
\end{hint}
\end{problem}
\begin{problem}
Prueba que el cualquier conjunto de 10 enteros positivos menores a 100, hay dos subconjuntos
sin elementos en común de números que tienen la misma suma
\begin{hint}
    El número de subcojuntos posibles es $2^10-1$ y la mayor suma de posible es $99+98+\cdots+91=945$
\end{hint}
\end{problem}
\vspace{0.1cm}

\begin{problem}
    Dentro de un cojunto de $n+1$ números distintos escogidos dentro de los números del $1$ al $2n$, existe uno que es múltiplo del otro
    \begin{hint}
        Todo número $k$ es de la forma $k=2^\alpha\cdots q$ con $q$ impar, por casillas hay dos números $k$ dentro de los escogidos que tienen la misma $q$(¿Por qué?)
    \end{hint}
\end{problem}
\vspace{0.1cm}
\begin{problem}
    Cada cuadrito de una cuadrículo de $3\times 7$ es coloreado de color blanco y negro. Muestre que en cualquier coloración siempre hay cuatro cuadritos del mismo color que son las esquinas de un rectángulo dentro de la cuadrícula
    \begin{hint}
        La columnas pueden quedar coloreadas de $8$ formas (haz los casos)
        \begin{center}
            \begin{asy}
            size(10cm);
                for (int i = 0; i <= 8; ++i) {
    draw((2i+1,0)--(2i+2,0)--(2i+2,3)--(2i+1,3)--cycle, gray);

}
size(10cm);
                for (int i = 0; i <= 8; ++i) {
    draw((2i+1,1)--(2i+2,1), gray);
draw((2i+1,2)--(2i+2,2), gray);
}

            \end{asy}
        \end{center}
 De esas $8$, nota que si dos se repiten entonces acabas, trata de llegar a ese caso de alguna forma   
    \end{hint}
\end{problem}


\clearpage

\bigskip

\section{Hints}
\Closesolutionfile{all-hints}
\begin{enumerate}
  \input{all-hints.out}
\end{enumerate}


\end{document}
