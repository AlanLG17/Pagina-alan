\documentclass[11pt]{scrartcl}
\usepackage[sexy]{evan}
\usepackage{amsthm}
\renewcommand{\proofname}{Prueba}
\usepackage{cancel}
\usepackage{xcolor}
\usepackage{answers}
\Newassociation{hint}{hintitem}{all-hints}
\renewcommand{\solutionextension}{out}
\renewenvironment{hintitem}[1]{\item[\bfseries #1.]}{}
\newcommand{\colorcancel}[2][black]{\renewcommand\CancelColor{\color{#1}}\cancel{#2}}
\title{Criterios de divisibilidad}


\author{AlanLG}
\date{Marzo 2024}

\begin{document}

\maketitle

\section{Lectura}
En esta lista trabajaremos propiedades elementales de teoría de números, la rama de las matemáticas que estudia el conjunto de los números enteros denotado por $\mathbb{Z}$:
\[\mathbb{Z}=\{\ldots,-2,-1,0,1,2,3,\ldots\}\]

Así como el conjunto de los naturales denotado por $\mathbb{N}$

\[\mathbb{N}=\{1,2,3,\ldots\}\]


\subsection{Divisibilidad}
Si $a$ y $b$ son dos número enteros ($a,b\in\mathbb{Z}$),decimos que $a$ divide a $b$ (ó bien $a\mid b$) si y solo si la división $\frac{a}{b}$ es entero, o bien
\[a\mid b \iff \frac{b}{a}\in \mathbb{Z}\]

También podemos decir que $a$ divide a $b$ si existe un entero $k$ tal que $b=a\cdot k$, o bien
\[a\mid b \iff \exists\hspace{0.1cm} k\in\mathbb{Z} : b=a\cdot k\]

Cuando $a$ \textbf{no divide} a $b$ lo denotamos por $a\nmid b$

\begin{ejemplo*}
    $7\mid 21$ pues $\frac{21}{7}=3\in \mathbb{Z}$
\end{ejemplo*}
\begin{ejemplo*}
    $20\mid 100$ pues $100=20\cdot(50)$
\end{ejemplo*}
\begin{ejemplo*}
    El 1 divide a todos los números (¿Por qué?)
\end{ejemplo*}
\subsection{Números primos}

\begin{proposition*}
[Definición de número primo]
    Decimos que un número entero positivo $p\neq \pm 1$ es primo si sus únicos divisores son $\pm1$ y $\pm p$
\end{proposition*}
\begin{definition*}
    Decimos que un número entero positivo diferente de $\pm 1$ es \textbf{Compuesto} si no es primo, a los número 1 y -1 se le llaman \textit{unidades}
\end{definition*}
\begin{ejemplo*}
    El número 31 es primo
\end{ejemplo*}
\begin{ejemplo*}
    El número 39 no es primo
\end{ejemplo*}

\begin{exercise*}
    Encuentra los primeros diez números primos
\end{exercise*}
\begin{ejemplo*}
    El único número primo que es par es el $2$ (¿Por qué?)
\end{ejemplo*}
\subsection{Criterios de divisibilidad}
En la siguiente lista introduciremos los criterios de divisibilidad, que sirven esencialmente para ver si un número divide a otro, muchos de ellos son intuitivos pero saberlos te puede facilitar un poco de cuentas.

\begin{criterio*}
    [Criterio de divisibilidad del 2]
    Que su último dígito sea divisible entre 2
\end{criterio*}
\begin{ejemplo*}
       $143\underline{4}$ es divisible entre 2 porque $4$ es divisible entre $2$
   \end{ejemplo*}

\begin{criterio*}
    [Criterio de divisibilidad del 3]
    Que la suma de sus dígitos sea divisible entre 3
\end{criterio*}
\begin{ejemplo*}
        2025 es divisible entre $3$ porque $2+0+2+5=9$ es divisible entre $3$
    \end{ejemplo*}
    \begin{ejemplo*}
        123456789 es divisible entre $3$ porque $1+2+3+4+5+6+7+8+9=45$ es divisible entre $3$
    \end{ejemplo*}
\begin{criterio*}
    [Criterio de divisibilidad del 4]
    Que el número formado por sus últimos dos dígitos sea divisible entre $4$
\end{criterio*}
\begin{ejemplo*}
    $333\underline{12}$ es divisible entre 4 porque 12 es divisible entre 4
\end{ejemplo*}
\begin{ejemplo*}
    $57\underline{04}$ es divisible entre 4 porque 4 es divisible entre 4
\end{ejemplo*}
\begin{criterio*}
    [Criterio de divisibilidad del 5]
    Que su último dígita sea $0$ ó $5$
\end{criterio*}
\begin{ejemplo*}
    $123\underline{0}$ es divisible entre 5 porque termina en 0
\end{ejemplo*}
\begin{ejemplo*}
    $343\underline{5}$ es divisible entre porque termina en 5
\end{ejemplo*}
\begin{criterio*}
    [Criterio de divisibilidad del 6]
    Que su último dígito sea par y la suma de sus dígitos sea divisible entre 3. Lo cúal es equivalente a que cumpla el criterio del 2 y del 3
\end{criterio*}
\begin{ejemplo*}
    12648 es divisible entre 6 porque termina en 8 y $1+2+6+4+8=21$ es divisible entre 3
\end{ejemplo*}

\begin{criterio*}
    [Criterio de divisibilidad del 8]
    Que el número formado por sus últimos 3 dígitos sea divisible entre 8
\end{criterio*}
\begin{ejemplo*}
    $45\underline{128}$ es divisible entre 8 porque 128 es divisible entre 8
\end{ejemplo*}
\begin{ejemplo*}
    $123\underline{000}$ es divisible entre 8 porque 0 es divisible entre 8
    \end{ejemplo*}
    \begin{ejemplo*}
    $2\overline{008}$ es divisible entre 8 porque 8 es divisible entre 8
\end{ejemplo*}
\begin{criterio*}
    [Criterio de divisibilidad del 9]
    Que la suma de sus dígitos sea divisible entre 9
\end{criterio*}
\begin{ejemplo*}
    126 es divisible entre 9 porque $1+2+6=9$ es divisible entre 9
\end{ejemplo*}
\begin{ejemplo*}
     123456789 es divisible entre 9 porque $1+2+3+4+5+6+7+8+9=45$ es divisible entre 9
\end{ejemplo*}
\begin{criterio*}
    [Criterio de divisibilidad del 10]
    Que su último dígito sea 0
\end{criterio*}
\begin{ejemplo*}
    $1000000\underline{0}$ es divisible entre 10 porque termina en 0
\end{ejemplo*}
\begin{ejemplo*}
\documentclass[11pt]{scrartcl}
\usepackage[sexy]{evan}
\usepackage{amsthm}
\renewcommand{\proofname}{Proof}


\usepackage{answers}
\Newassociation{hint}{hintitem}{all-hints}
\renewcommand{\solutionextension}{out}
\renewenvironment{hintitem}[1]{\item[\bfseries #1.]}{}
\raggedright

\title{sharkydevil}


\author{AlanLG}
\date{9 de abril del 2024}
\begin{document}
\maketitle
Consider the miquel point of quadrilateral $BCFE$, this is, the intersection of circles $(AEF)$ and $(ABC)$, we say this point is the $A$-sharkydevil point of $\triangle ABC$

    \begin{figure}[ht]
    \centering
\begin{asy}
import geometry;
   size(7cm);
   point B=(0,0); dot("B", B, SW);
   point C=(8,0); dot("C", C, SE);
   point A=(2,6.5); dot("A", A, N);
   filldraw(circle(A,B,C),opacity(0.05)+purple, purple);
 draw(A--B--C--cycle,purple);
 draw(incircle(A,B,C),purple);
  point I=incenter(A,B,C); dot ("I", I, S);
  point D=foot(I,B,C); dot ("D", D, S);
  point E=foot(I,A,C); dot ("E", E, S);
  point F=foot(I,A,B); dot ("F", F, S);
  draw(circle(A,E,F), lightred);
pair[] rrr=intersectionpoints(circle(E,F,I), circle(B,A,C));
  pair S = rrr[0]; dot("S", S, N);
   pair [] zzz=intersectionpoints(line(A,I), circle(A,B,C));
   pair M =zzz[0]; dot ("M", M, S);
   draw(E--F--D--cycle, lightblue);
   point P=foot(D, E, F); dot("P", P, N);
   draw(D--P, blue);
   draw(I--S, purple+dashed);
   draw(S--M, opacity(0.5)+red);
\end{asy}
\caption{The $A$-Sharkydevil point.}
  \label{fig:sharkydevil}
  \end{figure}

This point have many nice propierties, we are going to present the more known in the following table

\begin{theorem}[Most known propierties in Skarkydevil]
Using \Cref{fig:sharkydevil} notation
    
    \begin{enumerate}
        \item[1)]\label{1)} $SD$ bisects $\angle BSC$ then it passes through the midpoint $M$ of arc $\arc{BC}$
        \item[2)]\label{2)} $P$ be the foot from $D$ to $EF$ then $S$ is the inverse of $P$ wrt incircle
        \item[3)]\label{3)} $AS$ and $BC$ meet on $AI$ perpendicular through $I$
        \item[4)]\label{4)} $SD$ and $(AEF)$ meets on $A-$altitude
    \end{enumerate}
\end{theorem}

\newpage
\begin{figure}[ht]
    \centering
\begin{asy}
import geometry;
   size(12cm);
   
   point B=(0,0); dot("B", B, SW);
   point C=(8,0); dot("C", C, SE);
   point A=(2,6.5); dot("A", A, N);
   filldraw(circle(A,B,C),opacity(0.05)+purple, purple);
 draw(A--B--C--cycle,purple);
 draw(incircle(A,B,C),purple);
  point I=incenter(A,B,C); dot ("I", I, S);
  point D=foot(I,B,C); dot ("D", D, S);
  point E=foot(I,A,C); dot ("E", E, NE);
  point F=foot(I,A,B); dot ("F", F, SW);
  draw(circle(A,E,F), lightred);
pair[] rrr=intersectionpoints(circle(E,F,I), circle(B,A,C));
  pair S = rrr[0]; dot("S", S, NW);
   pair [] zzz=intersectionpoints(line(A,I), circle(A,B,C));
   pair M =zzz[0]; dot ("M", M, SE);
   draw(E--F--D--cycle, lightblue);
   point P=foot(D, E, F); dot("P", P, N);
   draw(D--P, blue);
   draw(I--S, purple+dashed);
   draw(S--M, opacity(0.5)+red);
  point V=intersectionpoint(line(A,S),line(B,C)); dot ("V", V, SW);
   draw(V--A,purple+dashed);
   draw(V--B,purple+dashed);
   point R= intersectionpoint(line(A,M), line(B,C)); dot ("A*", R, SE);
   draw(A--M, purple+dashed);
   draw(arc(M, length(M-B), 0, 180),purple+dashed);
   draw(circle(R,I,D), purple+dashed);
\end{asy}
\caption{$A$-Sharkydevil propierties}
  \label{fig:sharky2}
  \end{figure}

     \begin{claim}
         $SD$ bisects $\angle BSC$ then it passes through the midpoint $M$ of arc $\arc{BC}$
     \end{claim}
     \begin{proof}  
     invert wrt $(BIC)$, $A$ goes to $A^{*}=BC\cap AM$, $(AEF)$ goes to $(A^{*}ID)$ then so $S$ goes to $D$.\end{proof}

     \begin{claim}
         $P$ be the foot from $D$ to $EF$ then $S$ is the inverse of $P$ wrt incircle
     \end{claim}
 \begin{proof} invert wrt incircle, $(AEF)$ goes to $EF$, as this inversion swaps $(ABC)$ to the nine-point circle of $\triangle DEF$, then $P$ is the inverse of $S$\end{proof}
 \begin{claim}
     $AS$ and $BC$ meet on $AI$ perpendicular through $I$
 \end{claim}
 \begin{proof}  use Radical Axis Theorem on $(AFE), (BIC)$ and $(ABC)$ \end{proof}
  \begin{claim}
      $SD$ and $(AEF)$ meets on $A-$altitude
  \end{claim}
\begin{proof} Invert wrt $(BIC)$, let $R=(AEF)\cap SD$, $R^{*}=(IDA^{*})\cap SD$ so\[\angle MID=\angle MR^{*}A^{*}=\angle MAR\]consider $O$ the center of $(ABC)$, then $O$ goes to $O^{*}=MO\cap (BIC)$ then $BOCM$ is a rhombus and note that
 \[\angle MAO =\angle MO^{*}A^{*}=\angle O^{*}MA^{*}=\angle MID=\angle MAR \]
ergo $AU$ and $AO$ are isogonal\end{proof}
 
\newpage
 \begin{example}
     [Mexico TST 2024]

     Let $\Omega $ be the circumcircle of $\triangle ABC$ with incenter $I$. Let $M\neq A$ be the intersection of $AI$ and $\Omega $, and $D\in BC$ such that $ID\perp BC$. Let $E\in \Omega $ such that $AE\perp BC$. Let $N\neq I$ the intersection of $ID$ and $(BIC)$. Prove that $NE$ and $MD$ intersect at $\Omega$
 \end{example}
 \begin{figure}[ht]
    \centering
\begin{asy}
import geometry;
   size(10cm);
   point B=(0,0); dot("B", B, SW);
   point C=(8,0); dot("C", C, SE);
   point A=(2,6.5); dot("A", A, N);
   filldraw(circle(A,B,C),opacity(0.05)+red, red);
   point I=incenter(A,B,C); dot("I", I, S);
   draw(A--B--C--cycle,red);
   pair [] zzz=intersectionpoints(line(A,I), circle(A,B,C));
   pair M =zzz[0]; dot ("M", M, S);
   point D=foot(I,B,C); dot ("D", D, SE);
   point J=foot(A,B,C); 
   pair [] qqq=intersectionpoints(line(A,J), circle(A,B,C));
    pair E=qqq[0]; dot("E", E, SW);
   draw(circle(B,I,C), red);
   pair [] rrr=intersectionpoints(line(D,I), circle(I,B,C));
   pair N =rrr[0]; dot ("N", N, S);
   draw(I--N, red);
  pair S = intersectionpoint(line(M,D), line(E,N)); dot("S", S,NW);
   draw(M--S,red);
   draw(N--S, dashed+red);
   draw(A--M, red);
   markangleradiusfactor *= 0.5;
   markangle(A,M,S,blue);
   markangle(I,N,S,blue);
   markangle(A,E,S,blue);
   draw(A--E,red);
\end{asy}
\caption{Mexico TST/2024}
  \label{fig:mexicotst}
  \end{figure}
  We instantly note that $S=MD\cap (ABC)$ is the $A-$sharkydevil point, now note that 
  \[\angle AMS=\angle AES=\angle INS\]
  as $IS\parallel AE$ we are done; the last inequality follows by $IMNS$ being cyclic which is true because the inversion wrt $(BIC)$ send $\overline{IDN}$ to $IMNS$      


\end{document}