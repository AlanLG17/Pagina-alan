\documentclass[11pt]{scrartcl}
\usepackage[sexy]{evan}
\usepackage{amsthm}
\usepackage[helvetica]{quotchap} 

\title{aeiou}
\author{AlanLG}
\date{24 de Marzo del 2024}

\begin{document}

\maketitle

\epigraph{Solo existe un poder que todo poder garantice: ahondar hasta que cada cuestión se subatomice}
{\emph{Solitario}}


\begin{problem}[\href{}{OMMEB 2021}]
    Rogelio escribe una lista de los divisores positivos de $10!$ de menor a mayor. Luego multiplica los números
    que ocupan los lugares $10$ y $261$ de su lista. ¿Qué resultado obtiene Rogelio?
    

\end{problem}
\vspace{2.5cm}
\begin{problem}
    
    Si $x,y,z$ son números reales y 
    \[\frac{x}{y+z}+\frac{y}{z+x}+\frac{z}{x+y}=r\]
¿Cuáles son los valores posibles de $r$?
\end{problem}
\vspace{2.5cm}
\begin{problem}
Encuentra todos los enteros $n$ para los cuales $2^n+12^n+2011^n$ es un cuadrado
\begin{center}

\end{center}
\end{problem}
\vspace{2.5cm}
\begin{problem}
 
    [\href{https://artofproblemsolving.com/community/c1068820h2975916p26675216}{ITAMO 2000}] 
    Sea $ABCD$ un cuadrilátero convexo, y $\alpha=\angle DAB$, $\beta=\angle ADB$, $\gamma=\angle ACB$, $\delta= \angle DBC$ and $\epsilon=\angle DBA$. Si se cumple que $\alpha<\pi/2$, $\beta+\gamma=\pi /2$, y $\delta+2\epsilon=\pi$, prueba que $(DB+BC)^2=AD^2+AC^2$.

\end{problem}

\end{document}