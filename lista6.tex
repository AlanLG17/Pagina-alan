\documentclass[11pt]{scrartcl}
\usepackage[sexy]{evan}
\usepackage{amsthm}
\usepackage[helvetica]{quotchap} 

\title{asdfghjklñ}
\author{AlanLG}
\date{24 de Marzo del 2024}

\begin{document}

\maketitle

\epigraph{Un matemático, como un pintor o un poeta, es un creador de
patrones. Si sus patrones son más permanentes que los de otros
artistas, es porque están hechos de ideas.}
{\emph{G.H. Hardy}}

    
\begin{problem}


    ¿Cuánto vale el área sombreada?
\begin{center}
    \begin{asy}
        size(4cm);
        draw((0,0)--(3,0)--(2,3)--cycle);
        filldraw((0.97,1.46)--(1.89,0.8)--(2.63,1.11)--(2,3)--cycle,lightred);
        draw((0,0)--(2.63,1.11));
        draw((3,0)--(0.97,1.46));
        label("2", (2.51,0.64));
        label("3", (0.96,0.75));
        label("4",(1.63,0.27));
        label("?",(1.86,1.7));

    \end{asy}
\end{center}
\end{problem}
\vspace{0.1cm}

\begin{problem}
    Prueba que si $ab=cd$ entonces $a+b+c+d$ no es primo
\end{problem}
\vspace{0.1cm}

    \begin{problem}[\href{https://artofproblemsolving.com/community/c6h2339189p18843038}{OMM 2020}]

        Sea $n\ge 3$ un número entero. En un juego hay $n$ cajas en una matriz circular. Al principio, cada caja contiene un objeto que puede ser piedra, papel o tijera, de tal forma que no existen dos cajas adyacentes con el mismo objeto, y cada objeto aparece al menos en una caja.

\item Al igual que en el juego, la piedra vence a las tijeras, las tijeras vencen al papel y el papel vence a la piedra.

\item El juego consiste en mover objetos de una casilla a otra según la siguiente regla:
\begin{center}
\textit{Se eligen dos casillas adyacentes y un objeto de cada una de forma que sean diferentes, y trasladamos el objeto perdedor a la casilla que contiene el objeto ganador. Por ejemplo, si recogimos piedra del cuadro A y tijeras del cuadro B, movemos las tijeras al cuadro A.}
\end{center}
Demuestra que, aplicando la regla suficientes veces, es posible mover todos los objetos a la misma caja.
\end{problem}
\vspace{0.1cm}
\begin{problem}
 

Sea $n$ un entero positivo. Resuelve el siguiente sistema de ecuaciones
\[x_1+x_2^2+x_3^3+\cdots +x_n^n=n\]
\[x_1+2x_2+3x_3+\cdots+nx_n=\frac{n(n+1)}{2}

\end{problem}

\end{document}